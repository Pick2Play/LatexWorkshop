% /e:/Programming/Visual studio/Latex/test.tex
\documentclass[11pt]{article}
\usepackage[utf8]{inputenc}
\usepackage[T1]{fontenc}
\usepackage{lmodern}
\usepackage[margin=1in]{geometry}
\usepackage{microtype}
\usepackage{amsmath,amssymb}
\usepackage{hyperref}


\begin{document}

\begin{titlepage}

\begin{center}

    \vspace*{1cm}
    {\Huge \bfseries Entwicklung eines Stammdatenmoduls in Angular mit .NET-API-Anbindung unter Berücksichtigung von Usability und Datenintegrität \par}

    \large
    \vspace{1.5cm}
    \textbf{Hausarbeit}

    \vspace{1.5cm}
    \textbf{Hochschule Rhein-Waal}\\
    Fakultät: \textbf{Kommunikation und Umwelt}\\
    Studiengang: \textbf{Medieninformatik}\\
    Abgabe: \textbf{23.11.2025}\\

    \vspace{1cm}
    Workshop II: \textbf{Dozent}\\
    Vorgelegt von: \textbf{Andreju Douvletis}\\
    Matrikelnummer: \textbf{30233}\\
    
\thispagestyle{empty}
\end{center}
\end{titlepage}

\newpage

\tableofcontents
\pagenumbering{roman}
\setcounter{page}{2}

\newpage
\pagenumbering{arabic}
\setcounter{page}{1}

\section{Einleitung}

%Problemstellung und Motivation
%Zielsetzung der Arbeit
%Forschungsfrage und Abgrenzung
%Vorgehensweise und Aufbau der Arbeit

In dem ersten Abschnitt der Arbeit geht es um die Problemstellung, die Motivation, die Zielsetzung, die Abgrenzung und die Vorgehensweise. Das zu lösende Problem ist, dass die veraltete Weboberfläche zu einem erhöhten Arbeitsaufwand führt. 
Dieses Problem soll durch eine Modernisierung der Seite gelöst werden. Das alte Tool ähnelt einer Monolith Anwendung und soll deswegen durch die Modernisierung mehr in Richtung Microservices entwickelt werden.
Ziel des Prototyps ist, die Weboberfläche effizienter und benutzerfreundlicher für das Unternehmen zu machen. 
Die Forschungsfrage „Wie können UI- und UX-Designentscheidungen bei der Entwicklung eines Angular-basierten Stammdatenmoduls mit .NET-API-Anbindung zur Verbesserung der Usability beitragen?" soll die technische Umsetzung in den Vordergrund stellen. Dabei sollen die Wirkung der UI- und UX-Designentscheidungen von Quellen und spätere Usertests belegt werden. 

% Außerdem ist das alte Tool eine Monolith Anwendung weswegen die Modernisierung einen Teil der Anwendung zur Entstehung von mehreren Micorservices führen soll. -> Puh weiß ich nicht ob das so passt
% ohne spätere User Tests durchzuführen -> Das ist die große Frage

\section{Theoretische Grundlagen und Stand der Forschung}
%Grundlagen des Webdesigns
%Verwandte Arbeiten und bestehende Ansätze
%Relevante Theorien, Modelle oder Designprinzipien

Im zweiten Abschnitt geht es um die Grundlagen, verwandte Arbeiten und bestehende Ansätze und relevante Theorien, Modelle oder Designprinzipien. In Bezug auf die Grundlagen werden die Frameworks und nennenswerte Funktionalitäten kurz beschrieben. 
Bei verwandten Arbeiten und bestehenden Ansätzen wird nach ähnlichen wissenschaftlichen Beiträgen gesucht und der Fund beschrieben. Hier können dann Inspirationen unter anderem aufgeführt werden. 
Um die Grundlagen abzurunden, werden noch relevante Theorien, Modelle oder Designprinzipien aufgezeigt und erläutert. Dabei sind nicht nur die Methoden, sondern auch ihre Anwendung in der Arbeit relevant.


\section{Methodisches Vorgehen}
%Wahl und Begründung der Forschungsmethodik (Prototypischer Ansatz)
%Das Vorgehensmodell des Design Science Research
%Anwendung des Vorgehens auf die vorliegende Arbeit



\section{Konzeption des Artefakts}
%Anforderungsanalyse
%Abgeleitete Designprinzipien
%Konzeptionelles Modell und Systemarchitektur
%Geplante Funktionalitäten und Interaktionen

Im folgenden Abschnitt geht es um die Planung für die Entwicklung des Artefakts. Um das Artefakt gut planen zu können, sollte vorher eine Anforderungsanalyse angefertigt werden, um herauszufinden, welche Aspekte des Prototyps priorisiert werden sollten.
Es ist ebenfalls wichtig, zu planen, welche Designprinzipien oder Methoden während der Entwicklung verwendet werden sollen. Außerdem sind ein konzeptionelles Modell und ein Plan für die Systemarchitektur hilfreich, um bei der Umsetzung Zeit zu sparen. Zusätzlich zu der Anforderungsanalyse sollten hier auch geplante Funktionalitäten und Interaktionen dokumentiert werden. 

\section{Umsetzung des Artefakts}
%Technische Umsetzung / Prototyp-Entwicklung
%Gestalterische und funktionale Entscheidungen (Entwicklungsendscheidungen)
%Implementierungsdetails und Herausforderungen

Bei der technischen Umsetzung des Artefaktes geht es vor allem um die Entwicklungsentscheidungen, Implementierungsdetails und Herausforderungen. Die Begründung der Entwicklungsentscheidungen und das Aufzeigen von Alternativen sind für diesen Abschnitt entscheidend. 

\section{Evaluation}
%Evaluationskonzept und -methodik (Sind die Anforderungen Erfüllt?)
%Durchführung und Ergebnisse
%Diskussion der Ergebnisse

Im Evaluationsabschnitt geht es um den Vergleich der Anforderungen mit dem Endprodukt des Prototyps. Sind alle Anforderungen erledigt? Wurden alle Funktionalitäten implementiert? Diese und weitere Aspekte wie „was hätte man besser machen können?" werden in diesem Abschnitt thematisiert. 

% Hier könnte man auch noch erwähnen das Diskussionen mit den Nutzer und dem Vorgesetzten hier einfließen 

\section{Reflexion}
%Reflexion des Entwicklungsprozesses (Die Erkenntnisse)
%Abgeleitete Designprinzipien und Empfehlungen
%Limitationen und zukünftige Arbeiten

Im vorletzten Abschnitt wird der Entwicklungsprozess reflektiert. Hierbei werden die Design- und Entwicklungsentscheidungen nochmal beleuchtet und kritisch hinterfragt. Auch die Limitationen und Erkenntnisse werden für zukünftige Arbeiten beschrieben. 

\section{Fazit und Ausblick}
%Zusammenfassung der Ergebnisse
%Beantwortung der Forschungsfrage
%Ausblick auf zukünftige Entwicklungen

Im letzten Abschnitt wird das Ergebnis der Arbeit zusammengefasst und die Forschungsfrage beantwortet. Zum Schluss wird ein Ausblick auf die zukünftigen Entwicklungen formuliert.

% Hauptthema wird die Entwicklung des Frontends, Das Backend ist auch wichtig aber bezogen auf die Forschungsfrage ist das Frontend mehr gewichtet


\end{document}